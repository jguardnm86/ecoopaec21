\documentclass[parskip=full]{scrreprt}
\title{jGuard: Programming misuse-resilient APIs}
\subtitle{Artifacts for reproducibility}

\usepackage{tabularx}   % Tabellen, die sich automatisch der Breite anpassen
\usepackage{mathtools} % erweiterte Fassung von amsmath
\usepackage{amssymb}   % erweiterter Zeichensatz
\usepackage[binary-units]{siunitx}   % Einheiten
\usepackage{listings}
\usepackage{import}
\usepackage{xcolor}
\usepackage{tikz}
\usepackage{fancyvrb}
\usepackage{multirow}
\usepackage[nounderscore]{syntax} % Definitionen von Grammatiken
\usepackage{hyperref}

\usepackage{pgfplots}
\usepackage{pgf-pie}
\usepgfplotslibrary{statistics}

\pgfplotsset{compat=1.17}

\newcommand{\etal}{et al.\ }
\newcommand{\eg}{e.g.}

\newcommand{\syncDecl}[1]{<{\color{blue} \textbf{#1}}\label{gr:#1}>}
\newcommand{\syncRef}[1]{\hyperref[gr:#1]{\synt{{\color{blue} \textbf{#1}}}}}
\newcommand{\customSyncRef}[2]{\hyperref[gr:#1]{<\emph{\color{blue} \textbf{#2}}>}}
\newcommand{\todo}[1]{{\color{red} \textbf{TODO: } #1}}

\newcommand{\detectable}{\tikz[baseline=-0.5ex]\draw[green,fill=green,radius=3pt] (0,0) circle ;}%
\newcommand{\partiallyDetectable}{\tikz[baseline=-0.5ex]\draw[orange,fill=orange,radius=3pt] (0,0) circle ;}%
\newcommand{\notDetectable}{\tikz[baseline=-0.5ex]\draw[red,fill=red,radius=3pt] (0,0) circle ;}%

\newcommand\YAMLcolonstyle{\color{red}}
\newcommand\YAMLkeystyle{\color{black}}
\newcommand\YAMLvaluestyle{\color{blue}}

\lstset{
tabsize = 2, %% set tab space width
showstringspaces = false, %% prevent space marking in strings, string is defined as the text that is generally printed directly to the console
numbers = none, %% don't display line numbers by default
commentstyle = \color[HTML]{586e75}, %% set comment color
keywordstyle = \color{blue}, %% set keyword color
stringstyle = \color{red}, %% set string color
rulecolor = \color{black}, %% set frame color to avoid being affected by text color
basicstyle = \fontsize{7}{8} \ttfamily , %% set listing font and size
breaklines = true, %% enable line breaking
numberstyle = \tiny,
}

\lstdefinestyle{jguard}{
  language=Java,
  morekeywords={verified, require, initially, sets, instantiation, meta, when, returns, guard, reset, resets, state}
}

\lstdefinelanguage{CrySL}[]{Java}{
  morekeywords={ABSTRACT, SPEC, OBJECTS, EVENTS, ORDER, CONSTRAINTS, REQUIRES, ENSURES, REFINES, define, add, constraint},
  moredelim=[is][\textcolor{darkgray}]{\%\%}{\%\%},
  moredelim=[il][\textcolor{darkgray}]{§§}
}

\makeatletter
\usepackage[htt]{hyphenat}

% here is a macro expanding to the name of the language
% (handy if you decide to change it further down the road)
\newcommand\language@yaml{yaml}

\expandafter\expandafter\expandafter\lstdefinelanguage
\expandafter{\language@yaml}
{
  keywords={true,false,null,y,n},
  keywordstyle=\color{darkgray},
  sensitive=false,
  comment=[l]{\#},
  morecomment=[s]{/*}{*/},
  commentstyle=\color{purple}\ttfamily,
  stringstyle=\YAMLvaluestyle\ttfamily,
  moredelim=[l][\color{orange}]{\&},
  moredelim=[l][\color{magenta}]{*},
  moredelim=**[il][\YAMLcolonstyle{:}\YAMLvaluestyle]{:},   % switch to value style at :
  morestring=[b]',
  morestring=[b]",
  literate =    {---}{{\ProcessThreeDashes}}3
                {>}{{\textcolor{red}\textgreater}}1     
                {|}{{\textcolor{red}\textbar}}1 
                {\ -\ }{{\ -\ }}3,
}

\begin{document}

\maketitle
\setcounter{chapter}{1}

\section{Overview}

Our artifact consists of the following resources:

\begin{enumerate}
 \item A video explaining how to use jGuard, \url{https://youtu.be/tnQGMyMjZRA}.
 Please use this this video to test if jGuard works, it serves as a test for Kick-the-Tires.
 \item A public, anonymous repository, \url{https://github.com/jguardnm86/ecoopaec21}.
 This repository contains the MPS implementation for jGuard described in our paper.
 It also contains annotated code for BouncyCastle and core JDK classes which serve as the
 basis for our evaluation.
 \item For convenience, a prepared virtual machine (OVA file) and pre-compiled versions of our verified BouncyCastle implementation are available under \url{https://zenodo.org/record/5767812#.YbDYZ1Mo9MQ}.
\end{enumerate}

\section{Setup}

The video linked in the previous section explains how to setup jGuard. This section gives
an alternative textual description.
While our implementation is available as a VM, we recommend checking out jGuard from source if
possible.

\subsection{From sources}

Compiling jGuard requires version 2021.2 of JetBrains MPS\footnote{Available for download under \url{https://www.jetbrains.com/mps/download/previous.html}}, no additional dependencies are needed.
Clone our repository (\url{https://github.com/jguardnm86/ecoopaec21}) and load this
directory as an MPS project.
In MPS, first build the \texttt{apiSL} language implementing jGuard as described in our paper.
For this step, model checker warnings in MPS if any occur.
Next, build the \texttt{apiSL.runtime} model. At this point, all jGuard features are ready to be used in that MPS instance.

\subsection{Via the prebuilt VM}

We have prepared a \href{https://zenodo.org/record/5767812/files/jGuard_fixed.ova?download=1}{virtual machine} with MPS and jGuard installed.
To review jGuard this way, download the linked OVA file and import it into a suitable virtualization software (we were using VirtualBox version 6.1).

Launch in the VM. You should be taken to the desktop directly. If needed, login with the user and password \texttt{jguard}. Launch MPS from the side bar to open 

\section{Reproducing our evaluation}

After launching MPS with our implementation (and building the language if not using the prepared VM), the \texttt{sandbox} and \texttt{BouncyCastle} modules can be used to
reproduce our evaluation resultse for RQ1-a and RQ1-b, respectively.

\subsection{RQ1-a (Expressing guards against frequent misuses)}

The \texttt{documents/misusessummary.pdf} file in the linked GitHub repository gives an
overview of the most frequently occuring misuses in the MUBench dataset.
Further, it describes how we designed jGuard checks against these misuses.

As it is not possible to reliably replace core JDK classes on an unmodified VM, we rely
on custom tests in the \texttt{test} package in the \texttt{apiSL.sandbox} solution.
These tests construct instances of our verified classes and trigger a misuse and can be
run independently.
The verified classes will throw an exception when this misuse occurs.

\subsection{RQ1-b (Expressing guards against crypto misuses)}

You can use the \texttt{BouncyCastle} model in the MPS project to compile a verified implementation of the BouncyCastle JCA implementation.

We have used an incremental approach to verify BouncyCastle and only converted the classes
needing modifications to express guards against misuses.
To create a usable \texttt{jar} file from these files, replace the affected classes from an
unmodified BouncyCastle build. For your convenience, we have prepared \href{https://zenodo.org/record/5767812/files/verified_bc.jar?download=1}{a BouncyCastle jar with our changes} which can be used to run benchmarks or verify misuses.

For running CogniCrypt as a static analysis tool, we refer to the documentation of \href{https://github.com/CROSSINGTUD/CryptoAnalysis}{CogniCrypt\textsubscript{SAST}}.
As a data source, we used cryptographic misuses in \href{https://github.com/stg-tud/MUBench}{MUBench}. We identified cryptographic misuses as those using at least one class in
the \texttt{javax.crypto} package.
A detailed overview of relevant misuses and the guards we have designed against them is also given in the second section of the \texttt{documents/misusessummary.pdf} file in our repository.

To demonstrate how the \texttt{verified\_bc.jar} can be used, consider the \texttt{SimpleCryptoMisuse.java} file at the root of the linked repository.
This simple program misuses a JCA API, which can be verified as follows:

\begin{enumerate}
 \item Run \texttt{javac -cp benchmarks/poi-benchmark/bclibs/original\_bc.jar  SimpleCryptoMisuse.java} to compile the application.
 \item Run it against the original BouncyCastle implementation: \texttt{java -cp benchmarks/poi-benchmark/bclibs/original\_bc.jar:. SimpleCryptoMisuse}. No misuse is reported.
 \item Run it against the verified BouncyCastle implementation: \texttt{java -cp benchmarks/poi-benchmark/bclibs/verified\_bc.jar:. SimpleCryptoMisuse}, the misuse is reported.
\end{enumerate}

Similar to this synthetic example, cryptographic misuses reported in MUBench can be reproduced by downloading the source mentioned in MUBench and running it against the verified BouncyCastle implementation.

\subsection{RQ2 (Benchmarks)}

Relevant sources for our benchmarks are available in the \texttt{benchmarks} directory of our repository.
You can run our benchmarks as follows:

\begin{enumerate}
 \item Run \texttt{./gradlew jmh} in \texttt{benchmarks/bouncycastle-benchmark}.
 \item Run \texttt{./gradlew jmh} in \texttt{benchmarks/poi-benchmark}.
 \item Run \texttt{./gradlew jmh-benchmarks:jmh} in \texttt{benchmarks/java-jwt-bk}.
\end{enumerate}

Benchmark results are written into \texttt{build/results/jmh} in the folder the Gradle command was run in.

\end{document}
 
